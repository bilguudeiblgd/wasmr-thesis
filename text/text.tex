\chapter{Introduction}
The R programming language is ubiquitous for statistical computing and data analysis, offering powerful abstraction for data manipulation and visualization. Inspired by S-Language, the R language has been important part for industries working with statistics and been pioneer for introducing widespread tools currently used by the community such as DataFrame.

Traditionally, R is an interpretered language that runs on most major operating systems.
By interpretered language, we mean that the language source code is not translated to certain binary to be executed by processor
like compiled languages. Instead interpreter is a program, usually written in low-level languages such as C, C++  or Rust, which
takes the source code, does lexing and parsing to get AST and directly executes the program from its AST representation by evaluating
from top node \cite{crafting_interpreters,rbytecodebook}.

Nowadays, the industry has been expanding the environments where we can run our programs.
The interpreter program we discussed about, is compiled to bytecode, and then binary that
the processor can execute. However, with invention of WebAssembly, we can take this interpreter
and compile it to so-called WASM bytecode, and run it on web. This is exactly what webR \cite{webR} did.
This way, R code we write, can seamlessly run on most browsers.

Compiling the interpreter to WASM introduces certain challenges. 
Since the interpreter consists of a substantial amount of code, 
every R program executed in the browser must include the entire interpreter compiled to WASM. 
This increases the overall memory footprint and can negatively impact performance, 
particularly in resource-constrained environments.

Thus, this thesis explores direct compilation of R-like language to WebAssembly. A subset of R with type annotations is created
to make compilation possible to WASM, as it needs static typing. The thesis then
explores the challenges of mapping a dynamic high-level language to WASM, and finally evaluates the performance and correctness of the compiler.

The structure of the thesis follows this premise. Chapter 2 introduces the concepts
of R, WASM and previous works. Chapter 3 states the rationale for creating Typed R and introduces
the language itself, both formally and informally with examples. Chapter 4 describes the building of the compiler and runtime, and lastly Chapter 5 discusses
the evaluation of the compiler and the performance of the generated binary in comparison to standard R environment and WebR.
\section{Aim}
It's important to note this thesis focuses on exploring possibility and options of compiling R to WASM.
Creation of type system and annotations are a vehicle to reach there and not full focus of this thesis, as there
are better typing proposals \cite{typr2025,R_towards_type_system,empirical_r_types}.
Primarily the aim is to define a statically-typed subset of R, that is simple enough to be able to be compiled to WASM,
but also expressive enough to show R's core such as mathematical operations and vectors. Then show by implementing a working and an efficient
compiler for Typed R to WASM and evaluating the generation of WASM.