\chapter{Introduction}
The R programming language is ubiquitous for statistical computing and data analysis, offering powerful abstraction for data manipulation and visualization. Inspired by S-Language, the R language has been important part for industries working with statistics and been pioneer for introducing widespread tools currently used by the community such as DataFrame.

Traditionally, R is an interpretered language that runs on most major operating systems.
By interpretered language, we mean that the language source code is not translated to certain binary to be executed by processor 
like compiled languages. Instead interpreter is a program, usually written in low-level languages such as C, C++  or Rust, which 
takes the source code, does lexing and parsing to get AST and directly executes the program from its AST representation by evaluating 
from top node\cite{crafting_interpreters}. Figure~\ref{r-being-interpretted} shows overview of R code being interpretted.

\begin{figure}[h!]
    \includegraphics[width=370px]{images/r-ast-eval-process.png}
    \caption{Overview. \href{https://coolbutuseless.github.io/book/rbytecodebook/05-how-r-executes-code.html}{Taken from R Bytecode Book website}}
    \label{r-being-interpretted}
\end{figure}

Nowadays, the industry has been expanding the environments where we can run our programs. 
The interpreter program we discussed about, is compiled to bytecode, and then binary that 
the processor can execute. However, with invention of WebAssembly, we can take this interpreter
and compile it to so-called WASM bytecode, and run it on web. This is exactly what webR\cite{webR} did.
This way, R code we write, can seamlessly run on the web environment on most browsers.

Compiling the interpreter to WASM introduces certain challenges. 
Since the interpreter consists of a substantial amount of code, 
every R program executed in the browser must include the entire interpreter compiled to WASM. 
This increases the overall memory footprint and can negatively impact performance, 
particularly in resource-constrained environments.

Thus, this thesis explores direct compilation of R-like language to WebAssembly. I create
subset of R with type annotations, in order to make compilation possible to WASM, as it needs static typing. Then
explore the challenges of mapping a dynamic high-level language to WASM, and finally evaluate the performance and correctness of my compiler.


